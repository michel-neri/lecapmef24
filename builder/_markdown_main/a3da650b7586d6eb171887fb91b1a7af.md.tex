\markdownRendererDocumentBegin
\docpart{Développement : Méthode de Newton}\markdownRendererInterblockSeparator
{}\shortcutProposition{} : Soit $f:I = [a;b] \to \R$. Si\markdownRendererInterblockSeparator
{}\markdownRendererUlBeginTight
\markdownRendererUlItem $f(a)f(b) <0$,\markdownRendererUlItemEnd 
\markdownRendererUlItem $f$ dérivable sur $I$,\markdownRendererUlItemEnd 
\markdownRendererUlItem $f'>0$ sur $I$,\markdownRendererUlItemEnd 
\markdownRendererUlItem $f'$ str. croissante sur $I$,\markdownRendererUlItemEnd 
\markdownRendererUlItem $f$ est $\mathscr C^0$ sur $I$.\markdownRendererUlItemEnd 
\markdownRendererUlEndTight \markdownRendererInterblockSeparator
{}Alors l'équation $f(x)=0$ admet une unique solution $z \in I$. De plus, la suite $(x_n)_{n \in \N}$ définie par récurrence de la sorte\markdownRendererInterblockSeparator
{}$$ \begin{cases} x_0 \in \intOF{z}{b} \\ x_{n+1} = x_n - \frac{f(x_n)}{f'(x_n)} \end{cases} $$\markdownRendererInterblockSeparator
{}est bien définie et converge vers $z$.\markdownRendererInterblockSeparator
{}\markdownRendererEmphasis{Preuve}\markdownRendererInterblockSeparator
{}Note préliminaire : en appliquant le TVI on obtient que $z$ est l'unique zéro de $f$ sur $I$.\markdownRendererInterblockSeparator
{}On démontre que la suite est bien définie par récurrence.\markdownRendererInterblockSeparator
{}\underline{Initialisation} : Ok par construction.\markdownRendererInterblockSeparator
{}\underline{Hérédité} : Soit $n \in \N$. Supposons construit le terme $x_n$. Commençons par dire que comme $x_n \in [z;b] \subset [a;b]$, on peut composer $f$ comme $f'$ par $x_n$ sans souci. De plus, $f' > 0$ et donc on peut diviser par $f'(x_n)$, cela justifie la bonne définition du terme $x_{n+1}$.\markdownRendererInterblockSeparator
{}Ensuite, comme par hypothèse $f'$ est strictement croissante, $f$ est convexe et son graphe se situe donc au dessus de toutes ses tangentes. Plus particulièrement, $f'(x_n)(x-x_n) < f(x)$ qqsoit $x \in I \setminus \ens{x_n}$. Pour $x=z$ ($\neq x_n$ par hypothèse de récurrence), cela se formule $f'(x_n)(z-x_n) < f(x_n)$. En travaillant sur cette inéquation (attention à bien utiliser les hypothèses et la note préliminaire) on arrive à $z < x_n - \frac{f(x_n)}{f'(x_n)} = x_{n+1}$.\markdownRendererInterblockSeparator
{}Ensuite, un simple travail sur les signes permet d'affirmer que $-\frac{f(x_n)}{f'(x_n)} < 0$ et donc $x_{n+1} = x_n-\frac{f(x_n)}{f'(x_n)} < x_n$. Cela achève la preuve par récurrence.\markdownRendererInterblockSeparator
{}Nous avons donc montré la bonne définition de $\left(x_n\right)_n$. Au passage on a montré qu'elle était (strictement) décroissante. Comme elle est donc minorée (par $z$) et décroissante, elle converge vers un réel $l \in [z;b]$. Comme $f$ est continue et que $x_{n+1} = x_n - \frac{f(x_n)}{f'(x_n)}$, il vient par passage à la limite que $l = l - \frac{f(l)}{f'(l)}$. On en déduit que $f(l) = 0$, et donc par unicité de $z$ énoncée en préliminaire que $z=l$.\markdownRendererInterblockSeparator
{}\shortcutApplication{} : Résolution de $x^2=2$. On ré-écrit l'équation $x^2-2=0$. En montrant d'abord que $1 < \sqrt 2 < 2$ (en sachant à l'avance que $\sqrt 2$ est solution de $x^2-2=0$), appliquer la méthode de Newton à la résolution de $x^2-2=0$.\markdownRendererInterblockSeparator
{}La suite obtenue est dite « de Héron ». Faire le calcul à la main et écrire les premières approximations rationnelles à la main : $x_0=2$, $x_1=\frac{3}{2}$ et $x_2 = \frac{17}{12} \approx 1,41166$.\markdownRendererInterblockSeparator
{}Rq : L'étude de la convergence de cette suite peut se faire plus concrètement au lycée, c.f. BO Tle Spé, approfondissements.\markdownRendererInterblockSeparator
{}\markdownRendererEmphasis{Développement} : se fait en live.\markdownRendererInterblockSeparator
{}NB : Se renseigner sur la relaxation des hypothèses, cette question paraît naturelle vu l'armada de contraintes imposées.\markdownRendererDocumentEnd