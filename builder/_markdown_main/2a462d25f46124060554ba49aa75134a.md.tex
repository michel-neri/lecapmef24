\markdownRendererDocumentBegin
\docpart{L31 : Différents types de raisonnement en mathématiques}\markdownRendererInterblockSeparator
{}\markdownRendererSectionBegin
\markdownRendererHeadingOne{Hypothético-déductif}\markdownRendererInterblockSeparator
{}\markdownRendererUlBeginTight
\markdownRendererUlItem Méthode de Newton (appliquée à Héron, voir le développement).\markdownRendererUlItemEnd 
\markdownRendererUlItem $f \in \mathscr C^0([a;b])$, $f(x)=x$ solutions ?\markdownRendererUlItemEnd 
\markdownRendererUlEndTight \markdownRendererInterblockSeparator
{}
\markdownRendererSectionEnd \markdownRendererSectionBegin
\markdownRendererHeadingOne{Disjonction de cas}\markdownRendererInterblockSeparator
{}\markdownRendererUlBeginTight
\markdownRendererUlItem $n$ (imp)pair $\implies$ $n^2$ (im)pair.\markdownRendererUlItemEnd 
\markdownRendererUlItem $\max \ens{x,y} = \frac{1}{2} \left( x + y + |x-y| \right)$.\markdownRendererUlItemEnd 
\markdownRendererUlEndTight \markdownRendererInterblockSeparator
{}
\markdownRendererSectionEnd \markdownRendererSectionBegin
\markdownRendererHeadingOne{Analyse-synthèse}\markdownRendererInterblockSeparator
{}\markdownRendererUlBeginTight
\markdownRendererUlItem Résolution d'une $ax+by+c=0$.\markdownRendererUlItemEnd 
\markdownRendererUlItem $f = \text{impaire}+\text{paire}$.\markdownRendererUlItemEnd 
\markdownRendererUlEndTight \markdownRendererInterblockSeparator
{}
\markdownRendererSectionEnd \markdownRendererSectionBegin
\markdownRendererHeadingOne{Contraposition}\markdownRendererInterblockSeparator
{}\markdownRendererUlBeginTight
\markdownRendererUlItem $n^2$ (im)pair $\implies$ $n$ (im)pair.\markdownRendererUlItemEnd 
\markdownRendererUlEndTight \markdownRendererInterblockSeparator
{}
\markdownRendererSectionEnd \markdownRendererSectionBegin
\markdownRendererHeadingOne{Récurrence}\markdownRendererInterblockSeparator
{}\markdownRendererUlBeginTight
\markdownRendererUlItem Un exo « guidé » sur $A^n=PD^n^{-1}$.\markdownRendererUlItemEnd 
\markdownRendererUlItem Récurrence forte (mot non dit) pour le théorème d'existence d'une décomposition en produit de facteurs premiers, ou autre.\markdownRendererUlItemEnd 
\markdownRendererUlItem $\limite{n \to +\infty} q^n = +\infty$ ($q>1$), avec ING de Bernoulli.\markdownRendererUlItemEnd 
\markdownRendererUlEndTight \markdownRendererInterblockSeparator
{}
\markdownRendererSectionEnd \markdownRendererSectionBegin
\markdownRendererHeadingOne{Absurde}\markdownRendererInterblockSeparator
{}\markdownRendererUlBeginTight
\markdownRendererUlItem Proposition de 6\textsuperscript e sur parallélisme / perpendicularité à savoir montrer (renvoie probablement aux axiomes d'Euclide).\markdownRendererUlItemEnd 
\markdownRendererUlItem $\sqrt 2 \notin \Q$.\markdownRendererUlItemEnd 
\markdownRendererUlItem $| \P | = +\infty$.\markdownRendererUlItemEnd 
\markdownRendererUlEndTight \markdownRendererInterblockSeparator
{}
\markdownRendererSectionEnd \markdownRendererSectionBegin
\markdownRendererHeadingOne{Contre-Exemple}\markdownRendererInterblockSeparator
{}\markdownRendererUlBeginTight
\markdownRendererUlItem $n^2-n+41$ premier $\forall n$.\markdownRendererUlItemEnd 
\markdownRendererUlItem irrationnel $+$ irrationnel $=$ irrationnel (écrire « penser à $\pi$ » pour tendre la perche suivante : le jury se dira peut-être qu'il va piéger le candidat en lui demandant « ah oui et comment vous montrez que $\pi$ est irrationnel ? » : voir le développement sur l'irrationnalité de $\pi$).\markdownRendererUlItemEnd 
\markdownRendererUlItem « $\sqrt[10]{x}$ passe en dessous de $\ln x$ assez vite et reste en dessous à jamais. »\markdownRendererUlItemEnd 
\markdownRendererUlEndTight 
\markdownRendererSectionEnd \markdownRendererDocumentEnd