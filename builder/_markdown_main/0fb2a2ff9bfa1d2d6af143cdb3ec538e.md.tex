\markdownRendererDocumentBegin
\docpart{L31 : Théorème des valeurs intermédiaires}\markdownRendererInterblockSeparator
{}Contexte : Suites réelles.\markdownRendererInterblockSeparator
{}Prérequis : Notion de suite, de fonction et de continuité de fonction.\markdownRendererInterblockSeparator
{}\markdownRendererSectionBegin
\markdownRendererHeadingOne{Résultats préliminaires}\markdownRendererInterblockSeparator
{}\shortcutProposition{} : Toute suite monotone bornée est convergente.\markdownRendererInterblockSeparator
{}\shortcutDefinition{} : Suites adjacentes.\markdownRendererInterblockSeparator
{}\shortcutTheoreme{} : Deux suites adjacentes sont convergentes et convergent vers la même limite.\markdownRendererInterblockSeparator
{}
\markdownRendererSectionEnd \markdownRendererSectionBegin
\markdownRendererHeadingOne{TVI et corollaires}\markdownRendererInterblockSeparator
{}\shortcutTheoreme{} : Théorème des valeurs intermédiaires.\markdownRendererInterblockSeparator
{}\shortcutTheoreme{} : TVI monotone (théorème de la bijection).\markdownRendererInterblockSeparator
{}\shortcutCorollaire{} : Théorème de Bolzano (reformulation de $k$ entre $f(a)$ et $f(b)$ en $f(a)f(b) \leq 0$).\markdownRendererInterblockSeparator
{}
\markdownRendererSectionEnd \markdownRendererSectionBegin
\markdownRendererHeadingOne{Applications}\markdownRendererInterblockSeparator
{}\markdownRendererOlBeginTight
\markdownRendererOlItemWithNumber{1}Bête(s) exo(s) d'application.\markdownRendererOlItemEnd 
\markdownRendererOlItemWithNumber{2}Soit $f \in \mathscr C^0([0;1])$. Mq $f$ admet un point fixe (plus : généraliser à $[a;b]$)\markdownRendererOlItemEnd 
\markdownRendererOlItemWithNumber{3}Application à la résolution d'équations, méthodes exactes :\markdownRendererInterblockSeparator
{}\markdownRendererUlBeginTight
\markdownRendererUlItem Méthode de la dichotomie, implémentation Python / GeoGebra.\markdownRendererUlItemEnd 
\markdownRendererUlItem Méthode de Newton, voir développement correspondant.\markdownRendererUlItemEnd 
\markdownRendererUlEndTight \markdownRendererOlItemEnd 
\markdownRendererOlItemWithNumber{4}Théorème : Toute fonction polynôme de degré impair (et non constante) admet une racine rélle.\markdownRendererOlItemEnd 
\markdownRendererOlEndTight 
\markdownRendererSectionEnd \markdownRendererDocumentEnd